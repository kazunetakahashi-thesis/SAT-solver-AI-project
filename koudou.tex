%#!platex main.tex
%#BIBTEX pbibtex main

\section{行動}

ToDo:論理学の用語がほとんどわからないので少し調べる。
$R = 14$とする。二重否定則を採用する。
無限は絡まないので選択公理等はどうでもいいはず。

\subsection{「ならば」の変換}

SAT Solver に投げる命題に条件を落とし込むときに、
「ならば」で発想すると自然に記述できるように思う。
ところが「ならば」は SAT Solver には直接入力はできない。
そこで「ならば」の変換に必要な補題を用意する。

\begin{lem} \label{lem:land_lor}
 $a, b, c$を真か偽かの変数とする。以下が成立する。
 \[
 (a \land b) \lor c = (a \lor c) \land (b \lor c).
 \]
\end{lem}

\begin{proof}
 もっとかっこいい方法があるかもしれないが、私は$(a, b, c)$の
 真偽$8$パターンを確かめた。
 $a$が真のときと
 $(a, b, c) = (\text{真}, \text{真}, \text{偽})$のときは両辺真、
 残りは偽である。\qedhere
\end{proof}

\begin{cor} \label{cor:rightarrow}
 $a, b, c$を真か偽かの変数とする。以下が成立する。
 \[
  a \Rightarrow b = (\lnot a \lor b).
 \]
\end{cor}

\begin{proof}
 $a \Rightarrow b$は$\lnot a \lor (a \land b)$と同値である。
 あとは補題~\ref{lem:land_lor}を用いると、次式がしたがう。
 \[
  \lnot a \lor (a \land b) = (\lnot a \lor a) \land (\lnot a \lor b)
 = (\lnot a \lor b). \qedhere
 \]
\end{proof}

\subsection{行動した「地点」に関する入力}

ここでは、行動開始地点、行動終了地点、塗った地点に関する
基本的な条件を落とし込む。実際にフィールドを見て
それらを決定するプロセスは次小節にまわす。

まずは、行動地点に関する変数を導入する。

\begin{nota}
 $0 \leq i, j \leq R$とする。$k = 0, 1, 2 ,3$とする。
 \begin{enumerate}[1.]
  \item $s_{ij}$を「$i$行$j$列が行動開始地点ならば真、そうでないならば偽」とする。
  \item $e_{ij}$を「$i$行$j$列が行動終了地点ならば真、そうでないならば偽」とする。
  \item $d_{ijk}$を「$i$行$j$列方角$k$で塗ったならば真、そうでないならば偽」とする。
 \end{enumerate}
\end{nota}

最初に、当然だが、どこかの地点で行動開始、どこかの地点で塗り、どこかの地
点で行動終了していないといけない。これらの条件はそれぞれ
\begin{align*}
 (\text{Allow})_s &= \bigvee_{i = 0}^R \bigvee_{j = 0}^R s_{ij}, \\
 (\text{Allow})_e &= \bigvee_{i = 0}^R \bigvee_{j = 0}^R e_{ij}, \\
 (\text{Allow})_d &=
 \bigvee_{i = 0}^R \bigvee_{j = 0}^R \bigvee_{j = 0}^3 d_{ijk}
\end{align*}
と表せる。

次に、これも当然なことだが、$2$つ以上のマスが行動開始地点であってはならな
い。このことは SAT Solver に明示的に入力される必要がある。
つまり、全ての$(i, j) \neq (k, l)$に対し、
$\lnot (s_{ij} \land s_{kl})$である。すなわち、
\[
 (\text{Forbid})_s = 
 \bigwedge_{(i, j) \neq (k, l)} \lnot
 (s_{ij} \land s_{kl})
\]
である。これを SAT の入力に合うように変形すると、
\begin{align*}
 (\text{Forbid})_s &= \bigwedge_{(i, j) \neq (k, l)} (\lnot
 s_{ij} \lor \lnot s_{kl}) 
\end{align*}
である。これは$e, d$も同様である。すなわち、
\begin{align*}
 (\text{Forbid})_e &= \bigwedge_{(i, j) \neq (k, l)} (\lnot
 e_{ij} \lor \lnot e_{kl}), \\
 (\text{Forbid})_d &= \bigwedge_{(i, j, k) \neq (l, m, n)} (\lnot
 e_{ijk} \lor \lnot e_{lmn})
\end{align*}
である。

次に、移動はせいぜい$1$マスしかできないことを条件に加える。
$k, l$はそれぞれ$0$以上$R$以下とする。
しばらく、点$(k, l)$から
$0$または$1$しか離れていないマス$(i, j)$について$\bigvee$をとった
ものを$\bigvee_{i, j}$と表記することにする。素朴には
\[
 \bigvee_{i, j} = \bigvee_{i = k-1}^{k+1} \bigvee_{i = l-1}^{l+1} 
\]
のことだが、$(k, l)$が端のマスだったら$(i, j)$はフィールドの
外かもしれない。その部分は実装の際に適宜外すことにする。

さて、行動開始地点が点$(k, l)$のとき、行動終了地点が
せいぜい$1$マスしか離れていないことは、任意の$(k, l)$に対して
\[
 s_{kl} \Rightarrow \left( \bigvee_{i, j} e_{ij} \right)
\]
であると書ける。補題~\ref{cor:rightarrow}より、これは
\[
 \lnot s_{kl} \lor \left( \bigvee_{i, j} e_{ij} \right)
\]
と同値である。$(k, l)$について$\bigwedge$をとって、
\[
 (\text{start-end}) = \bigwedge_{k = 0}^R \bigwedge_{l = 0}^R 
 \left(\lnot s_{kl} \lor \left( \bigvee_{i, j} e_{ij} \right) \right)\
\]
が求める条件である。

次に塗った位置と方角についての条件を書く。つまり「塗った位置は、開始地点
か終了地点である」ということである。すなわち、任意の$(i, j, k)$に対し、
\[
 d_{ijk} \Longrightarrow s_{ij} \lor e_{ij}
\]
である。補題~\ref{cor:rightarrow}より、これは
\[
 \lnot d_{ijk} \lor (s_{ij} \lor e_{ij})
\]
と同値である。すなわち、条件は
\[
 (\text{drawpoint}) =
 \bigwedge_{i = 0}^{R} \bigwedge_{j = 0}^R \bigwedge_{k = 0}^3
 (\lnot d_{ijk} \lor s_{ij} \lor e_{ij})
\]
である。

\begin{rem}
 ここまでの制約をまとめると、以下の通りである。
 \begin{itemize}
  \item 行動開始地点と行動終了地点がただ$1$つ存在する。
  \item 行動開始地点と行動終了地点はせいぜい$1$マスしか離れていない。
  \item 塗った地点と塗った方向がただ$1$つ存在する。
  \item 塗った地点は、行動開始地点または行動終了地点である。
 \end{itemize}
\end{rem}

\subsection{塗った地点の制約}



% Local Variables:
% mode: yatex
% coding: utf-8
% TeX-master: "main.tex"
% End:

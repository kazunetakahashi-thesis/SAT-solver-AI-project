%#!platex main.tex
%#BIBTEX pbibtex main

\section{行動}

ToDo:論理学の用語がほとんどわからないので少し調べる。
$R = 14$とする。二重否定則を採用する。
無限は絡まないので選択公理等はどうでもいいはず。

\begin{lem} \label{lem:land_lor}
 $a, b, c$を真か偽かの変数とする。以下が成立する。
 \[
 (a \land b) \lor c = (a \lor c) \land (b \lor c).
 \]
\end{lem}

\begin{proof}
 もっとかっこいい方法があるかもしれないが、私は$(a, b, c)$の
 真偽$8$パターンを確かめた。
 $a$が真のときと
 $(a, b, c) = (\text{偽}, \text{真}, \text{真})$のときは両辺真、
 残りは偽である。\qedhere
\end{proof}

\begin{cor} \label{cor:rightarrow}
 $a, b, c$を真か偽かの変数とする。以下が成立する。
 \[
  a \Rightarrow b = (\lnot a \lor b).
 \]
\end{cor}

\begin{proof}
 $a \Rightarrow b$は$\lnot a \lor (a \land b)$と同値である。
 あとは補題~\ref{lem:land_lor}を用いると、次式がしたがう。
 \[
  \lnot a \lor (a \land b) = (\lnot a \lor a) \land (\lnot a \lor b)
 = (\lnot a \lor b). \qedhere
 \]
\end{proof}

\begin{nota}
 $0 \leq i, j \leq R$とする。
 $s_{ij}$を「$i$行$j$列が行動開始地点ならば真、そうでないならば偽」とする。
 $e_{ij}$を「$i$行$j$列が行動終了地点ならば真、そうでないならば偽」とする。
\end{nota}

まず、ごく当然なことだが、$2$つ以上のマスが行動開始地点であってはならな
い。このことは SAT Solver に明示的に入力される必要がある。
つまり、全ての$(i, j) \neq (k, l)$に対し、
$\lnot (s_{ij} \land s_{kl})$である。すなわち、
\[
 (\text{Forbid})_s = \bigwedge_{i = 0}^R \bigwedge_{j = 0}^R \lnot
 (s_{ij} \land s_{kl})
\]
である。これを SAT の入力に合うように変形すると、
\begin{align*}
 (\text{Forbid})_s &= \bigwedge_{i = 0}^R \bigwedge_{j = 0}^R (\lnot
 s_{ij} \lor \lnot s_{kl}) 
\end{align*}
である。これは$e$も同様である。すなわち、
\begin{align*}
 (\text{Forbid})_e &= \bigwedge_{i = 0}^R \bigwedge_{j = 0}^R (\lnot
 e_{ij} \lor \lnot e_{kl}) 
\end{align*}
である。

次に、移動はせいぜい$1$マスしかできないことを条件に加える。
$k, l$はそれぞれ$1$以上$R$以下とする。
しばらく、点$(k, l)$から
$0$または$1$しか離れていないマス$(i, j)$について$\bigvee$をとった
ものを$\bigvee_{i, j}$と表記することにする。素朴には
\[
 \bigvee_{i, j} = \bigvee_{i = k-1}^{k+1} \bigvee_{i = l-1}^{l+1} 
\]
のことだが、$(k, l)$が端のマスだったら$(i, j)$はフィールドの
外かもしれない。その部分は実装の際に適宜外すことにする。

さて、行動開始地点が点$(k, l)$のとき、行動終了地点が
せいぜい$1$マスしか離れていないことは、任意の$(k, l)$に対して
\[
 s_{kl} \Rightarrow \left( \bigvee_{i, j} e_{ij} \right)
\]
であると書ける。補題~\ref{cor:rightarrow}より、これは
\[
 \lnot s_{kl} \lor \left( \bigvee_{i, j} e_{ij} \right)
\]
と同値である。$(k, l)$について$\bigwedge$をとって、
\[
 (\text{start-end}) = \bigwedge_{k = 0}^R \bigwedge_{l = 0}^R 
 \left(\lnot s_{kl} \lor \left( \bigvee_{i, j} e_{ij} \right) \right)
\]
が求める条件である。

% Local Variables:
% mode: yatex
% coding: utf-8
% TeX-master: "main.tex"
% End:

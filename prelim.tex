%#!platex main.tex
%#BIBTEX pbibtex main

\section{準備}

二重否定則を採用する。
無限は絡まないので選択公理等はどうでもいいはず。

\subsection{「ならば」の変換}

SAT Solver に投げる命題に条件を落とし込とむきに、
「ならば」で発想すると自然に記述できるように思う。
ところが「ならば」は SAT Solver には直接入力はできない。
そこで「ならば」の変換に必要な補題を用意する。

\begin{lem} \label{lem:land_lor}
 $a, b, c$を真か偽かの変数とする。以下が成立する。
 \[
 (a \land b) \lor c = (a \lor c) \land (b \lor c).
 \]
\end{lem}

\begin{proof}
 もっとかっこいい方法があるかもしれないが、私は$(a, b, c)$の
 真偽$8$パターンを確かめた。
 $a$が真のときと
 $(a, b, c) = (\text{真}, \text{真}, \text{偽})$のときは両辺真、
 残りは偽である。\qedhere
\end{proof}

\begin{cor} \label{cor:rightarrow}
 $a, b$を真か偽かの変数とする。以下が成立する。
 \[
  a \Rightarrow b = (\lnot a \lor b).
 \]
\end{cor}

\begin{proof}
 $a \Rightarrow b$は$\lnot a \lor (a \land b)$と同値である。
 あとは補題~\ref{lem:land_lor}を用いると、次式がしたがう。
 \[
  \lnot a \lor (a \land b) = (\lnot a \lor a) \land (\lnot a \lor b)
 = (\lnot a \lor b). \qedhere
 \]
\end{proof}

% Local Variables:
% mode: yatex
% coding: utf-8
% TeX-master: "main.tex"
% End:

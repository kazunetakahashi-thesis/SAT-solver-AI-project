%#!platex main.tex
%#BIBTEX pbibtex main

\section{準備}

論理式を記述する際に使われる記号を述べ,以降で必要な考察をする.
本レポートを通し,命題論理では二重否定則を採用する.
「無限」は全く議論する必要がないので,
選択公理は採用しても採用しなくとも同じである.

\subsection{定義}

\begin{nota}
 $x, y$を命題とする.
 \begin{enumerate}[1.]
  \item 「$x$の否定」を$\lnot x$と書く.
  \item 「$x$かつ$y$」のことを$x \land y$と書き,
        「$x$または$y$」のことを$x \lor y$と書く.
  \item 「$x$ならば$y$」のことを$x \Rightarrow y$と書く.
 \end{enumerate}
\end{nota}

\begin{nota}
 $I$を集合とし,$i \in I$に対し$x_i$は命題とする.
 \begin{enumerate}[1.]
  \item 「任意の$i \in I$に対し$x_i$である」のことを
        \[
         \bigwedge_{i \in I} x_i
        \]
        と書く.$I = n \in \N$が有限集合であるとき,
        \[
         \bigwedge_{i \in I} x_i = x_0 \land x_1 \land \dots \land x_{n-1}
        \]
        である.
  \item 「$i \in I$が存在し$x_i$である」のことを
        \[
         \bigvee_{i \in I} x_i
        \]
        と書く.$I = n \in \N$が有限集合であるとき,
        \[
         \bigvee_{i \in I} x_i = x_0 \lor x_1 \lor \dots \lor x_{n-1}
        \]
        である.
 \end{enumerate}
\end{nota}

\begin{defn}
 \begin{enumerate}[1.]
  \item $x$が真か偽の値を取る変数であるとき,$x$を単に\emph{変数}という.
  \item $x$を変数とする.このとき,$x$の\emph{リテラル}とは,
        命題$x$,および,命題$\lnot x$のことである.
  \item \emph{節}とは,有限個のリテラル$l_1, \dots, l_n$を用いて
        \[
         \bigvee_{i = 1}^n l_n = l_1 \lor l_2 \lor \dots \lor l_n
        \]
        と表される命題である.この節の長さを$n$と定義する.
 \end{enumerate}
\end{defn}

\begin{defn}
 $x_1, \dots, x_n$を変数とする.
 \begin{enumerate}[1.]
  \item SAT Solver の\emph{入力}は,
        $x_1, \dots, x_n$のリテラルから作られた有限個の節
        $C_1, \dots, C_m$を用いて
        \[
        X = \bigwedge_{i=1}^m C_i = C_1 \land C_2 \land \dots \land C_m
        \]
        と表される命題である.$C_1, \dots, C_m$の長さがすべて$k$であると
        き,$X$を{\futoji $k$-SAT 問題}という.
  \item $X$を入力した際の SAT Solver の\emph{出力}は,
        次の(a), (b)のいずれかである.
        \begin{enumerate}[(a)]
         \item 偽である.この場合は,$X$を真にする$(x_1, \dots, x_n)$の
               真偽値の組が存在しないことを表す.
         \item 真である.この場合は,$X$を真にする$(x_1, \dots, x_n)$の
               真偽値の組が存在することを表す.
               この場合,更に追加で,$X$を真にする$(x_1, \dots, x_n)$の
               真偽値の組を$1$つ返す.
        \end{enumerate} 
 \end{enumerate}
\end{defn}

\subsection{「ならば」の変換}

SAT Solver に投げる命題に条件を落とし込とむきに,
「ならば」で発想すると自然に記述できるように思う.
ところが「ならば」は SAT Solver には直接入力はできない.
そこで「ならば」の変換に必要な補題を用意する.

\begin{lem} \label{lem:land_lor}
 $a, b, c$を変数とする.以下が成立する.
 \[
 (a \land b) \lor c = (a \lor c) \land (b \lor c).
 \]
\end{lem}

\begin{proof}
 もっとかっこいい方法があるかもしれないが,私は$(a, b, c)$の
 真偽$8$パターンを確かめた.
 $a$が真のときと
 $(a, b, c) = (\text{真}, \text{真}, \text{偽})$のときは両辺真,
 残りは偽である.\qedhere
\end{proof}

\begin{cor} \label{cor:rightarrow}
 $a, b$を変数とする.以下が成立する.
 \[
  a \Rightarrow b = (\lnot a \lor b).
 \]
\end{cor}

\begin{proof}
 $a \Rightarrow b$は$\lnot a \lor (a \land b)$と同値である.
 あとは補題~\ref{lem:land_lor}を用いると,次式がしたがう.
 \[
  \lnot a \lor (a \land b) = (\lnot a \lor a) \land (\lnot a \lor b)
 = (\lnot a \lor b). \qedhere
 \]
\end{proof}

% Local Variables:
% mode: yatex
% coding: utf-8
% TeX-master: "main.tex"
% End:

%#!platex main.tex
%#BIBTEX pbibtex main

\begin{abstract}
 SamurAI coding 2016--17 において,索敵処理を SAT Solver で実装する.
 本レポートでは,制約をいかに変数と論理式に変換し SAT Solver
 の入力に直したかを解説する.
\end{abstract}

\section{はじめに}

\subsection{SamurAI Coding 2016--17 の概要}

SamurAI Coding は情報処理学会が主催するプログラミングコンテストである.
2012 年度以降,毎年度開催されている.その開催理由は,公式サイトに
以下の通りに記されている~\cite{koushiki}.

\begin{quotation}
情報処理学会は若い世代から将来第一線の研究者や開発者になりうる,また世
界市場を舞台に活躍できる人材を育てることを目的として,(中略)
今年度も SamurAI Coding 2016--17 を開催いたします.ゲームをテーマにした人
工知能およびプログラミングのスキルを競い,(中略)
参加エンジニア・プログラマはその能力が世界で通用する
か本コンテストを通じて試すことができます.
\end{quotation}

本年度は任天堂のゲーム「スプラトゥーン」を模したゲームを
題材とした AI プログラミングコンテストである~\cite{koushiki-rules}.

\begin{quotation}
SamurAI Coding はプログラマが AI プログラミングの技倆を競うコンテストで
す. コンテストでは,それぞれ3人のサムライからなるふたつの軍団が \textit{SamurAI
3x3 2016} (サムライ スリー オン スリー 2016) という名のゲームを戦い,各
AI プログラムは同一軍団の3人のサムライを制御します.
\end{quotation}

SamurAI Coding 2016--17 のルールは
公式ドキュメント~\cite{rules}に記されている.
このレポートでも,このドキュメントの用語は断り無く使用する.

\subsection{SAT Solver との関係}

SAT Solver は,論理式を与えると,
それを真にする変数の組み合わせが存在するかしないかを調べ,
仮に存在するならばその変数の組み合わせを$1$通り与えるというものである.
詳しくは次節で定義する.
$k \geq 3$のとき,$k$-SAT 問題は NP 完全問題であることは有名である.
近年では,複数の工夫により SAT Solver の高速化が進んでいる.
また,SAT Solver を使う側の視点でも恩恵がある.
問題を一度論理式に直せば,その問題の効率的な解法がわからずとも
SAT Solver に与えれば高速に解が得られる.
汎用的な様々な応用も研究されている.
これらの事情は~\cite{weko_169441_1},~\cite{Knuth201512}に詳しい.

今回は,SAT Solver の AI プログラミングコンテストへの応用を目指す.
SamurAI Coding 2016--17 の AI を組む上で様々な処理が必要であるが,
そのうち,索敵部分の処理を SAT Solver で実装する構想を練った.
本レポートでは,敵の行動とフィールドの
様々な制約をいかに変数と論理式に変換し SAT Solver
の入力に直したかを解説する.

数学的な工夫として,
「直前の行動開始地点と行動終了地点が違う」という条件を SAT Solver の入
力に直そうとする際に,$O\left(2^{R^2}\right)$個の節を入力する必要に迫られたが,
別の制約を駆使して$O(R^2)$個へと簡約化した.
後述の\eqref{eq:p_r_2}と\eqref{eq:p_r_2_prime}を参照されたい.

\subsection{数理科学広域演習 IV との関係}

数理科学広域演習 IV で,本実装を解説するドキュメント,
すなわちこのレポートを整備することとした.
ドキュメントを整備した目的は,次のとおりである.

通常のプログラムでは,様々なデータ構造やアルゴリズムを
利用するため,ある程度手だれた者であれば
処理内容に見当がつくであろう.しかし,
本構想を実装したプログラムは,
入力から論理式をつくり, SAT Solver へ与え解かせ,その出力を処理するだけである.
また,その論理式へ至る思考の多くも,初等的とはいえ数学的な内容となっている.
したがって,完成されたプログラムだけを読んでも,せいぜい論理式を
把握できるだけで,「意味」を理解するのは極めて困難であると考えられる.
そこで,解説をするドキュメントを整備することとした.

2016 年度の数理科学広域演習 IV では,「ドキュメント」という
テーマで各自が活動した.私は,実習時間を使って,
本実装を考案し本ドキュメントを整備した.

% Local Variables:
% mode: yatex
% coding: utf-8
% TeX-master: "main.tex"
% End:

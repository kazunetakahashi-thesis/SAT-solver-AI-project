%#!platex main.tex
%#BIBTEX pbibtex main

\section{定義}

\begin{nota}
 $x, y$を命題とする.
 \begin{enumerate}[1.]
  \item 「$x$の否定」を$\lnot x$と書く.
  \item 「$x$かつ$y$」のことを$x \land y$と書き,
        「$x$または$y$」のことを$x \lor y$と書く.
  \item 「$x$ならば$y$」のことを$x \rightarrow y$と書く.
 \end{enumerate}
\end{nota}

\begin{nota}
 $I$を集合とし,$i \in I$に対し$x_i$は命題とする.
 \begin{enumerate}[1.]
  \item 「任意の$i \in I$に対し$x_i$である」のことを
        \[
         \bigwedge_{i \in I} x_i
        \]
        と書く.$I = n \in \N$が有限集合であるとき,
        \[
         \bigwedge_{i \in I} x_i = x_0 \land x_1 \land \dots \land x_{n-1}
        \]
        である.
  \item 「$i \in I$が存在し$x_i$である」のことを
        \[
         \bigvee_{i \in I} x_i
        \]
        と書く.$I = n \in \N$が有限集合であるとき,
        \[
         \bigvee_{i \in I} x_i = x_0 \lor x_1 \lor \dots \lor x_{n-1}
        \]
        である.
 \end{enumerate}
\end{nota}

\begin{defn}
 \begin{enumerate}[1.]
  \item $x$が真か偽の値を取る変数であるとき,$x$を単に\emph{変数}という.
  \item $x$を変数とする.このとき,$x$の\emph{リテラル}とは,
        命題$x$,および,命題$\lnot x$のことである.
  \item \emph{節}とは,有限個のリテラル$l_1, \dots, l_n$を用いて
        \[
         \bigvee_{i = 1}^n l_n = l_1 \lor l_2 \lor \dots \lor l_n
        \]
        と表される命題である.この節の長さを$n$と定義する.
 \end{enumerate}
\end{defn}

\begin{defn}
 $x_1, \dots, x_n$を変数とする.
 \begin{enumerate}[1.]
  \item SAT Solver の\emph{入力}は,
        $x_1, \dots, x_n$のリテラルから作られた有限個の節
        $C_1, \dots, C_m$を用いて
        \[
        X = \bigwedge_{i=1}^m C_i = C_1 \land C_2 \land \dots \land C_m
        \]
        と表される命題である.$C_1, \dots, C_m$の長さがすべて$k$であると
        き,$X$を{\futoji $k$-SAT 問題}という.
  \item $X$を入力した際の SAT Solver の\emph{出力}は,
        次の(a), (b)のいずれかである.
        \begin{enumerate}[(a)]
         \item 偽である.この場合は,$X$を真にする$(x_1, \dots, x_n)$の
               真偽値の組が存在しないことを表す.
         \item 真である.この場合は,$X$を真にする$(x_1, \dots, x_n)$の
               真偽値の組が存在することを表す.
               この場合,更に追加で,$X$を真にする$(x_1, \dots, x_n)$の
               真偽値の組を$1$つ返す.
        \end{enumerate} 
 \end{enumerate}
\end{defn}

% Local Variables:
% mode: yatex
% coding: utf-8
% TeX-master: "main.tex"
% End:
